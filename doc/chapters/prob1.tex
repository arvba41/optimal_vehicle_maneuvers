\chapter[Particle model optimization]{Friction-limited particle with and without rate-limitation}
Investigating the friction-limited particle and the friction-limited particle with rate-limited direction control with a comparison. 

The optimization problems follow the following general equation structure: 
\begin{align}
    & \underset{u}{\text{Min}}
    & & J\\
%
    & \text{subject to} 
    & & f_u(u) <= 0\\
%
    &&& f_o(x,u) <= 0 \\
%
    &&& \dot x = f(x,u),\\
%
    &&& x_0,\ x_f,
\end{align}
where $u$ is(are) the optimization variable(s), $J$ is the cost function, $f_u(u) <= 0$ and $f_o(x,u) <= 0$ denote the constraints on the $u$ and the states $x$, $\dot x = f(x,u)$ are the ODEs or dynamic constraints, and $x_0,\ \&\ x_f$ are the boundry conditions. 

\begin{description}
    \item[Opimization variables:] the longitudinal and lateral forces $u_x$ and $u_y$ on the vehicle. 
    \item[Cost function:] to minimize time $t$.
    \item[Constraints:] The forces on the vehicle are limited elliptically, i.e., \begin{equation*}
        u_x^2 + u_y^2 \leq (\mu m g)^2.
    \end{equation*}
    The force limit on the vehicle for the rate-limited direction control is given by \begin{align*}
        u_1^2 \leq (\mu m g)^2.
    \end{align*}\par The obstacle is given by the following equation:
    \begin{align*}
        \left(\frac{x - X_a}{R_1}\right)^n + \left(\frac{y}{R_2}\right)^n \geq 1
    \end{align*}
    \item[Vehicle model:] The friction-limited particle is given as follows: \begin{align*}
        \dot x &= v_x, \\
        \dot x &= v_x, \\
        m\,\dot v_x &= u_x,\\
        m\,\dot v_x &= u_y. \\
    \end{align*}
    The friction-limited particle with rate-limited direction control is given as follows: \begin{align*}
        \dot x &= v_x, \\
        \dot x &= v_x, \\
        m\,\dot v_x &= u_1\,\cos\left(\delta\right),\\
        m\,\dot v_y &= u_1\,\sin\left(\delta\right), \\
        \dot \delta &= u_2. \\
    \end{align*}
    \item[Miscellaneous constraints:] In order to ensure that the solution is within the desired operating space, certain miscellaneous constraints are included such as, \begin{align*}
        x_0 \leq x \leq x_f\\
        y_0 \leq y \leq y_f\\
        y_{min} \leq y \leq y_{max} \\
        0 \leq v_x 
    \end{align*} 
    For the rate-limited direction control model, the steering angle and steering rate is also constrained, i.e., \begin{align*}
        |\delta| \leq \delta_{max}, \\
        |\dot\delta| \leq \dot\delta_{max}, \\
    \end{align*}
\end{description}

The model, optimization, and obstacle parameters are presented in Tables~\ref{tab:flp_mdlparams},~\ref{sub@tab:flp_optparams}, and~\ref{sub@tab:flp_obsparams}, respectively. 

\begin{table}[h!]
    \begin{subtable}[h]{0.3\textwidth}
        \centering
        \begin{tabular}{c|c}
            - & value \\
            \hline
            $m$ & 500\,kg\\
            $g$ & 9.8\,m/s\textsuperscript{2}\\
            $\mu$ & 0.8\\
        \end{tabular}
        \caption{Model parameters}
        \label{tab:flp_mdlparams}
    \end{subtable}
    \hfill
    \begin{subtable}[h]{0.3\textwidth}
        \centering
        \begin{tabular}{c|c}
            - & value \\
            \hline
            $x_0$ & 0\,m\\
            $x_f$ & 100\,m\\
            $y_0$ \& $y_f$ & 1\,m\\
            $v_x$ & 40\,km/h\\
            $v_y$ & 0\,km/h
        \end{tabular}
        \caption{Initial parameters}
        \label{tab:flp_optparams}
    \end{subtable}
    \hfill
    \begin{subtable}[h]{0.3\textwidth}
        \centering
        \begin{tabular}{c|c}
            - & value \\
            \hline
            $X_a$ & 50\,m\\
            $R_1$ & 2\,m\\
            $R_2$ & 1.5\,m\\
            $n$ & 6\\
        \end{tabular}
        \caption{Obstacle parameters}
        \label{tab:flp_obsparams}
    \end{subtable}
    \caption{Model, optimization, and obstacle parameters.}
    \label{tab:temps}
\end{table}

The optimal control problem (OCP), was solved with direct multiple-shooting with 40 control intervals for the optimization using Matlab and CasADi. The ODE was solved using the fixed-step Runge-Kutta 4 integration method. 

The results of the optimization are presented in 

\begin{figure}[h!]
    \centering
    \includegraphics{figures/flp_avoid.pdf}
    \caption{Obstacle avoidance trajectory for the friction- and rate-limited particle model.}
    \label{fig:res_traj_p1}
\end{figure}

\pagebreak

\begin{table}[h]
    \centering
    \begin{tabular}{c|c|c}
        - & min $t$ & min $-v_f$ \\
        \hline
        $t$ & 3.83\,s & 3.94\,s \\
        $v_x(t_f)$ & 147.59\,km/h & 146.03\,km/h\\
    \end{tabular}
    \caption{Results for friction-limited and rate-limited particle model.}
    \label{tab:prob1_res}
\end{table}

From Table~\ref{tab:prob1_res}, it is clear that the friction-limited particle model (fric-PM) is slightly faster than the rate-limited particle model (rate-PM). 
This is because in rate-PM the rate of change of direction of the particle is limited and as a result, the ability of the vehicle to make a sharp turn is restricted and thus takes a longer time to complete the maneuver. 
This is visible in the control signals and state variables for the optimal trajectory shown in Figure~\ref{fig:prob1_res_detail}. 

Additional constraints and initial values for the firc-PM and rate-PM are presented in Table~\ref{tab:const_p1}. 

\begin{table}[h]
    \centering
    \begin{subtable}[h]{0.4\textwidth}
        \begin{tabular}{c|c||c|c}
            \multicolumn{2}{c||}{fric-PM} & \multicolumn{2}{c}{rate-PM}\\
            \hline
            $y_{max}$ & 5 & $\delta_{max}$ & $\pi/2$ \\
            - & - & $\dot\delta_{max}$ & $\pi/6$ 
        \end{tabular}
        \caption{Constraints.}
        \label{tab:const_p1a}
    \end{subtable}
    % \hfill
    \begin{subtable}[h]{0.4\textwidth}
        \begin{tabular}{c|c||c|c}
            \multicolumn{2}{c||}{fric-PM} & \multicolumn{2}{c}{rate-PM}\\
            \hline
            $v_x$ & 40\,km/h & $v_x$ & 40\,km/h
        \end{tabular}
        \caption{Initial conditions.}
        \label{tab:const_p1a}
    \end{subtable}
    \caption{Constraints for the fric-PM and rate-PM.}
    \label{tab:const_p1}
\end{table}

\begin{figure}[h!]
    \includegraphics{figures/flp_avoid_detailed.pdf}
    \caption{Detailed optimal trajectory states and control inputs for friction- and rate-limited particle models.}
    \label{fig:prob1_res_detail}
\end{figure}

\noindent\fbox{%
    \parbox{\textwidth}{%
        \textbf{Some reflections}: \newline
        The fric-PM can converge faster than the rate-PM in some cases. 
        Since 'ipopt' is used to solve the optimization problem, fric-PM is more sensitive toward the initializations (initial guesses). 
        Therefore, additional constraints may be necessary to improve convergence.
        Furthermore, the rate-PM does not have this problem and thus can have a lower computational time.
        However, the computational time can get long with 'bad' initialization conditions. 
        The convergence of this model seems better than the fric-PM.
        
        It is worth mentioning that the terms 'improved convergence' and 'better convergence' mean the ability of the solver to produce an 'optimal solution found' even with ridiculous guesses. However, one should take care not to fall into local minima pits. 
    }%
}

\section{Code}
The code for this problem can be found at 